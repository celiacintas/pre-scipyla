\documentclass[bigger]{beamer}
\usepackage[utf8]{inputenc}

%\usepackage{emerald}
\usepackage[T1]{fontenc}
\usepackage[absolute,overlay]{textpos}
%\usepackage[bookmarks=false,pdffitwindow]{hyperref}
\usepackage{fancybox}
%\usepackage{graphicx}
%\usepackage{minted}
\usepackage{animate}
%\usepackage{graphics}
\usepackage{xcolor}
\usetheme{Berkeley-structure}
\definecolor{ao}{rgb}{0.0, 0.5, 0.0}
\usecolortheme[named=ao]{structure}

\begin{document}
\logo{  \includegraphics[scale=0.2,keepaspectratio]{Imagenes/logos.png}}
\title{Showing some of the goodies: pandas, scikit-learn and matplotlib\\[0.5cm]}
\subtitle{Celia Cintas \\[0.2cm] Pre-SciPyLa 2015 \\[0.3cm]
\includegraphics[scale=0.2]{Imagenes/scipy.png}}
\date{}
\begin{frame}
\titlepage
\end{frame}
\section{Introduction}
%\begin{frame}[fragile]{ The Speaker}
%	\texttt{\$ whoami}
%		\begin{itemize}
%			\item 
%		\end{itemize}
%\end{frame}
\begin{frame}{\$ whoami}
\begin{itemize}
    \item Estudiante de Doctorado en Ciencias de la Computación (UNS).
    \item Docente en UNPSJB.
    \item Trabajo en proyectos del CENPAT-CONICET y el Lab de Cs de las Imágenes.
    \item Co-Organizadora of SciPyCon Argentina 2013 y 2014.
\end{itemize}
\begin{figure}
        \includegraphics[scale = 0.25]{Imagenes/uni.png}
\end{figure}
\end{frame}

\section{Qu\'e es Python?}

\begin{frame}{¿Qu\'e es Python ?}
    Python es un lenguaje de programación dinámico de propósito general. Algunas de sus características son:
    \begin{itemize}
        \item Fácil de aprender.
        \item Es todo terreno. 
        \item Es libre, con un ecosistema libre.
        \item Interactivo, pueden verse resultados rápidamente.
        \item Puede comunicarse con otros lenguajes fácilmente.(R, Fortran, C, C++, LaTeX)
        \item Argentina tiene la comunidad hispanohablante más grande.
        \item Muy bien documentado.
    \end{itemize}
% \begin{figure}[h]
%   \begin{center}
%       \includegraphics[scale = 0.4]{venn.png}
%   \end{center}
    %\caption{Landmarks de tipo II, imagen obtenida de ~\cite{Gonzalez2011} }
%\end{figure}
%Hacer diagrama de venn
\end{frame}

%---------------------------------------------------------------------------
\begin{frame}{Qu\'e podemos hacer con Python?}
 \begin{figure}[h]
%   \begin{center}
        \includegraphics[scale = 0.25]{Imagenes/quehacer5.png}
%   \end{center}
    %\caption{Landmarks de tipo II, imagen obtenida de ~\cite{Gonzalez2011} }
\end{figure}
\end{frame}

\begin{frame}{Qu\'e podemos hacer con Python? (Cont.)}

 \begin{figure}[h]
%   \begin{center}
        \includegraphics[scale = 0.2]{Imagenes/quehacer2.png}
%   \end{center}
    %\caption{Landmarks de tipo II, imagen obtenida de ~\cite{Gonzalez2011} }
\end{figure}
\end{frame}



%---------------------------------------------------------------------------
\begin{frame}{Qu\'e podemos hacer con Python? (Cont.)}

 \begin{figure}[h]
%   \begin{center}
        \includegraphics[scale = 0.18]{Imagenes/quehacer3.png}
%   \end{center}
    %\caption{Landmarks de tipo II, imagen obtenida de ~\cite{Gonzalez2011} }
\end{figure}
\end{frame}


%---------------------------------------------------------------------------
\begin{frame}{Qu\'e podemos hacer con Python? (Cont.)}

 \begin{figure}[h]
%   \begin{center}
        \includegraphics[scale = 0.22]{Imagenes/quehacer4.png}
%   \end{center}
    %\caption{Landmarks de tipo II, imagen obtenida de ~\cite{Gonzalez2011} }
\end{figure}
\end{frame}


%---------------------------------------------------------------------------
\section{Cómo el ecosistema de Python Científico puede ayudarme en mi trabajo?}
\begin{frame}{Flujo de trabajo básico}
 \begin{figure}[h]
%   \begin{center}
        \includegraphics[scale = 0.6]{Imagenes/0-.png}
%   \end{center}
    %\caption{Landmarks de tipo II, imagen obtenida de ~\cite{Gonzalez2011} }
\end{figure}
\end{frame}
%---------------------------------------------------------------------------
\section{Cómo el ecosistema de Python Científico puede ayudarme en mi trabajo?}
\begin{frame}{Flujo de trabajo básico (cont.)}
 \begin{figure}[h]
%   \begin{center}
        \includegraphics[scale = 0.6]{Imagenes/1-.png}
%   \end{center}
    %\caption{Landmarks de tipo II, imagen obtenida de ~\cite{Gonzalez2011} }
\end{figure}
\end{frame}

%---------------------------------------------------------------------------
\section{Instalación}
\begin{frame}[fragile]{Cómo se instala todo esto?!}
    \begin{itemize}
        \item Linux
        \begin{itemize}
            \item \begin{verbatim} # apt-get install ... \end{verbatim}
            \item \begin{verbatim} # pacman -S ... \end{verbatim}
            \item \begin{verbatim} # pip install ...\end{verbatim}
        \end{itemize}
        \item Multiplataforma
        \begin{itemize}
            \item Anaconda
            \item Canopy
            \item Python(x,y)
        \end{itemize}
        \item Mac OS
        \begin{itemize}
            \item SciPy SuperPack
        \end{itemize}
    \end{itemize}
\end{frame}
%---------------------------------------------------------------------------
\begin{frame}[fragile]{Instalando Anaconda ...}
     \begin{figure}[h]
%   \begin{center}
        \includegraphics[scale = 0.22]{Imagenes/instalacion.png}
%   \end{center}
    %\caption{Landmarks de tipo II, imagen obtenida de ~\cite{Gonzalez2011} }
\end{figure}
\end{frame}
%---------------------------------------------------------------------------
\begin{frame}{Instalando Anaconda ... (Cont.)}
Qué es lo que acabamos de instalar?
\begin{itemize}
    \item Spyder.
    \item Ipython Notebook.
    \item Ipython QTConsole.
    \item Ipython.
    \item Anaconda Console.
    \item Wakari.
\end{itemize}
\end{frame}
%---------------------------------------------------------------------------
\section{Muy bonito todo .. pero vamos a los bifes}
\begin{frame}{A programar ...}
 \begin{figure}[h]
%   \begin{center}
        \includegraphics[scale = 0.25]{Imagenes/xkcd.png}
%   \end{center}
    %\caption{Landmarks de tipo II, imagen obtenida de ~\cite{Gonzalez2011} }
\end{figure}
\end{frame}

\end{document}

